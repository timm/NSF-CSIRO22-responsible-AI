\section{Schedule} 


 \begin{wraptable}{r}{3.35in}
{ \footnotesize
\begin{tabular}{|l|l|l|l|}\cline{2-4}    
\multicolumn{1}{c|}{~}      &\multicolumn{3}{c|}{Year} \\
\multicolumn{1}{l|}{~}       &1 & 2 & 3  \\\hline  
{\bf Goal2:} run faster                   &   x  &  &   \\\hline 
{\bf Goal1:} better post-attack performance         &    x  &       &   \\\hline 
{\bf Goal3:} test against adaptive adversaries        &      & x      &   \\\hline  
{\bf Goal4:}  test in numerous domains                &      & x     &       \\ \hline
{\bf Goal5:} inspect the decision boundaries           &      & x      &  x     \\ \hline
BPC work (broadening participation in computing)  &   x &  x    & x     \\ \hline
\end{tabular} } 
\caption{Timetable for this work. }\label{when}
\end{wraptable}
Table~\ref{when} shows a three-year plan from this work.
Note that   will explore {\bf GOAL2} first since, if successful, this will speed up everything else.
Also,  we   explore {\bf GOAL5} last since everything up till
then has the potential to change the generated decision boundaries.

Lastly, 
one  of the benefits of  NSF funding
is the opportunity to work on  broadening participating in computing (BPC).  Our BPC plans are discussed in \S\ref{bpc}. As seen
  in our timetable, BPC will be an on-going task through-out the work.

% \subsection{Other Details}
% This NSF solicitation has certain required
% headings. Much of the information needed for those headings
% has already been presented. That said, we aggregate that information here.

% \subsubsection{Results from prior CSSI awards}

% This proposal is based on prior results generated from
% Elements: Can Empirical SE be Adapted to Computational Science? Award \#1931425
% (PI= Dr. Tim Menzies, NCState. 2019-2022).
% Problems with COVID   meant that we could
% not attend enough CSSI meetings to contact more groups.
% That said, using telecommunication tools,
% we spend much time interviewing CSS developers looking for their specific project needs.
% Samples of results from that CSSI were presented in 
% \fig{health} and 
% Table~\ref{casestudy}. 
 

% \subsubsection{Cyberinfrastructure Plans}

% As said in our introduction,  
% a unique feature of this proposal is its scope.
% All the tools generated by this  project  will  be  placed  on-line  in  a  free-to-access  Github  repository  and  made  available  to  the  CSS community via an open source license.
% The software produced here will be applicable to  {\em any} CSS project using an open source repository (and we know at least 700 such projects). In our concept of operation 
% for any project  registered  with {\IT},
% whenever developers commit code, they automatically receive an issue report commenting on the current and future health of their software, along with advice on how to improve that future state.  
 
%  \subsubsection{Measurable Outcomes}
% In terms of measuring the success of this work,
% those  
%  \textcolor{red}{{\bf measurable outcomes}} are listed in red (see above).
 
%  Also,  all the code developed as part of this work will be  released as open-source software in GitHub, under an MIT license.  Included in those
% packages will be the data used to certify the scripts as well as {\em RQn.sh} files containing executable scripts to reproduce (e.g.) RQ1.



%   \subsubsection{Management and Coordination Plan}
  
%   As said in our introduction, it is not enough to merely advertise some service and expect the community to use it. Much of the funds requested by this work is for  NC State researchers  to 
%   analyze as many CSS projects as possible. With these preliminary results in hand, they will approach CSS projects offering commentaries on their software (derived via our data miners)  and free consulting services on how to improve those projects. By  enticing CSS project members with results from their own projects, we anticipate growing a large user base amongst the CSS community.
  
 

 

%  \subsection{Related work in the Explanation Literature}
 

% When discussing this work with colleagues, they sometimes comment that
% xPLAIN is more a ``planner'' (on what to change) than an   ``explaination''
% device.
%  To those colleagues,
%   we  reply that there
%  is
% much precedent in the AI literature
% for connecting planning to explaination.
% In their systematic literature review on AI and explanation, 
%  Violane et al.~\cite{vilone2020explainable} use terminology that we find to be synonymous with our
%   plans that recommend what to change in order to most improve a system.  Violane et al. report is that
%   one of the 
%  most widespread use of explanations
%  in AI is to  find ways to change a
%  models behaviour; e.g. to debug a system in order
%  to stop some bug reoccurring).  
 
%  Other researchers make analogous conclusions.
% Adadi and  Berrada~\cite{Adadi18} list
% four main motivations for building
% explanation systems: 
% (a)~{\em explaining to justify} decisions made
% by some  model;
% (b)~{\em explaining to control} a system,
% allowing its debugging and the identification of potential flaws;
% (c)~{\em  explain to improve} 
% predictive performance and/or efficiency;
% and 
% (d)~{\em explaining to discover} novel  relationships and patterns in the data. As shown by the examples
% below, our ``planning as explanation'' methods  address motivation (b,c,d).

% Furthermore, the Violane et al. review
% lists no less than 37
% terms connected
% to current AI papers discussing  ``explaination''\footnote{
% Algorithmic transparency, 
% actionability, 
% causality, 
% completeness, 
% comprehensibility, 
% cognitive, 
% correctability, 
% effectiveness, 
% efficiency, 
% explicability, 
% explicitness, 
% faithfulness, 
% intelligibility, 
% interactivity, 
% interestingness, 
% interpretability, 
% informativeness, 
% justifiability, 
% mental fit, 
% monotonicity, 
% persuasiveness, 
% predictability, 
% reversibility, 
% robustness, 
% satisfaction, 
% scrutability / diagnosis, 
% security, 
% selection/simplicity, 
% sensitivity, 
% simplification, 
% soundness, 
% stability, 
% transparency, 
% transferability and, 
% understandability}.
% Semi-supervised planning relates to at 
% least two of the terms in the  
% Violane et al. survey: actionability and  effectiveness (since both these terms comment on if a plan can be deployed and (after that) how well the plan actually works.

  
  
 \section{Intellectual Merit and Broader Impact}
 \subsection{Intellectual Merit}
 The     OMNI-1  results suggests that  prior work made a premature and 
 incorrect conclusion about the value of diversity-based defences (against adversarial learning).  
   The use of ensembles
 has its
defenders~\cite{kariyappa2019improving,biggio2010multiple,DBLP:conf/iclr/TramerKPGBM18,smutz2016tree,kantchelian2016evasion} and it detractors~\cite{zhang2020decision,zhang2018gradient,he2017adversarial,DBLP:conf/iclr/TramerKPGBM18,DBLP:journals/corr/PapernotMG16}. 
Here, 
 we argue here before we can use ensembles to defend against attackers, we need to change the way we build and use ensembles.  Specifically, our ensembles based their  conclusions come from exploring the 
 {\em far corners of a very large ensemble}, rather than just the center
of a   small ensemble. 


  Prior work   concluded
    that    different  defending
  learners generate the  same decision boundary
  (the region separating malicious from benign inputs). This is highly
  undesirable since it  mean
   attackers can also learn that structure, {\em even if the attacker does not know
  the learner being used by the defender}.  Yet if this were always so, then OMNI-1 
  would not have worked.  The possibility, to be explored here with OMNI-1, is that  if we jump around a very large range of defense options, we can effectively defend against a wide range of adversarial attacks. 
  
  

 


\subsection{Broader Impacts} 

The more the international community connects (via software), the more that community
is susceptible to attacks. 
 This work will increase America’s ability for industrial and academic innovators to secure their own
 work.  
This in an important area of research since there  is an increasing reliance of
computational methods in all aspects of our society. 

 
 
As to other broader impacts, NC State's Computer Science department has a strong record of studying research issues related to gender bias~\cite{pullreq_17}, barriers faced by women ~\cite{ford2016paradise}, and methods for broadening participation~\cite{selfies} in the context of software engineering. 
 This work will inform the curriculum  and lecture notes of the various NC State NSF-funded REUs (research experience for undergraduates) as well as graduate
SE classes (taught by PI Menzies).  
PI Menzies will continue his established tradition of graduating research students for historically under-represented groups.
Also, funds from this work will be used to support students attending the Grace Hopper conference,
and the Richard Tapia Celebration  of Diversity in Computing.



Finally,  a (small) portion of this grant would be allocated to support Broadening Participation in Computing (BPC) work
(see next section).

 
 
 

\subsection{BPC Work: Broader Participation in Computer Science} \label{bpc}



  PI Menzies is a member of
his department's Broadening Participation Committee (BPC) that actively seeks to:
(a)~{\em Understand} what factors make computer science
more (or less) attractive to underrepresented groups;
(b)~{\em Educate} faculty, staff, and students on how different behaviors   effect diversity, quality and inclusiveness;
(c)~{\em Increases} the percent of students who identify as women; 
and (d)~{\em Evaluate} the success of his department's BPC team in broadening
participating.
 

 
\subsection{Dissemination of Knowledge}

All the code developed as part of this work will be  released as open-source software in GitHub, under an MIT license.  Included in those
packages will be the data used to certify the scripts as well as {\em RQn.sh} files containing executable scripts to reproduce (e.g.) RQ1.

  As to papers, PI Menzies frequently publishes as top-ranked international scientific venues and it is  anticipated that this work will generated multiple papers at
such venues. 

Also, this work will generated much  material (tools, scripts, data sets) that can be utilized by other research teams. 
For  a decade, PI Menzies has  lead-by-example in the open science community (the PROMISE project and the ROSE initiative) which takes care
not only to package and distribute research code but also publish papers and tutorials on that material.  
 
