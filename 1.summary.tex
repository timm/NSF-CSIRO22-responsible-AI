


\begin{nsfsummary} 
\begin{center}
{\bf \TITLE}\\\vspace{1mm}
% Tim Menzies,  IEEE Fellow, NC State
 \end{center}
  
\noindent

\noindent{\bf  OVERVIEW}

We propose a new tool, called \mbox{OMNI-2} that explores a core issue in adversarial attacks
(do different learners return the same decision boundary?)
while also allowing security practitioners and researchers to build more robust models against  white-box   adversarial evasion attacks.

 Many  malware and intrusion detection systems use machine learning-based   detection models. Prior work has shown that such models are susceptible to adversarial evasion attacks where inputs are    crafted by intelligent malicious adversaries  in order make the defender
(e.g. a neural net) misclassify malicious inputs as ``benign''. 
To recognize an adversarial attack,
 a defender must configure a  
learner.  
Using   {\bf {\em multi-objective geometry tricks}}, OMNI-2 can  explore the configuration of a large number of possible    defenders, then cache
100s of {\bf {\em nearly-optimal}}, but {\bf {\em unexpectedly different}}
ones.
By jumping randomly between all those configurations, it is possible to   prevent an adversary from finding,
then exploiting,
patterns in the defenses.
 

\noindent{\bf  INTELLECTUAL  MERIT}

The  success of   OMNI-1   suggests that  prior work made a premature and 
 incorrect conclusion about the value of diversity-based defences (against adversarial learning).  
  Prior work   concluded
    that    different  defending
  learners generate the  same decision boundary
  (the region separating malicious from benign inputs). This is highly
  undesirable since it  mean
   attackers can also learn that structure, {\em even if the attacker does not know
  the learner being used by the defender}.  Yet if this were always so, then OMNI-1 
  would not have worked.  The possibility, to be explored here with OMNI-1, is that  if we jump around a very large range of defense options, we can effectively defend against a wide range of adversarial attacks. 
% This work revisits the value of
% using   ensembles for defending against attackers. The use of such ensembles
%  has its
% defenders and it detractors.
% Here, 
%  we argue here before we can use ensembles to defend against attackers, 
%  it is vital to  change the way we build and use ensembles.  Our ensembles based their  conclusions come from exploring the 
%  {\em far corners of a very large ensemble}, rather than just the center
% of a   small ensemble. 
 



\noindent{\bf BROADER IMPACTS}
 
 
 This work will increase America’s ability for industrial and academic innovators to secure their own
 work.  
This in an important area of research since there  is an increasing reliance of
computational methods in all aspects of our society. 
But the more the international community connects (via software), the more that community
is susceptible to attacks. 
With the growing reliance on information technology, cybercrime is a serious threat to the economy, military and other industrial sectors.
In 2019, the damage cost caused by malware and cybercrime exceeded a trillion dollars. A 2020 study by Accenture reports that cybercrime will cost US \$5.2 trillion over the next five years. Alarming, that cost is growing: the annual average cost to organization of malicious software has grown 29\% in the last year. 



 
 
 In terms of other broader impact,
 this work will inform the curriculum  and lecture notes of the various NC State NSF-funded REUs (research experience for undergraduates) as well as graduate
SE classes (taught by PI Menzies). 
PI  Menzies  will  continue  his  established  tradition  of  graduating  research  students  from  historically under-represented  groups.   
Funds from this work will be used to support students attending the Grace Hopper conference,
and the Richard Tapia Celebration  of Diversity in Computing.

Further,  a (small) portion of this grant would be allocated to support Broadening Participation in Computing (BPC) work. NC  State’s  Computer  Science  department  has  a  strong  record  of  studying research issues related to gender bias,  barriers faced by women, and methods for broadening participation    in  the  context  of  software  engineering.   PI  Menzies  is  a  member  of  his  department’s Broadening Participation Committee (BPC) that actively seeks to:  (a)~Understand hat factors make computer science more (or less) attractive to underrepresented groups; (b)~Educate faculty, staff, and students on how different behaviors effect diversity, quality and inclusiveness; (c)~Increase the percent of students who identify as women; and (d)~Evaluate the success of his department’s BPC team in broadening participating

 

 
 
 
 
 
 

\noindent{\bf KEYWORDS}

Data Science, ML and AI,
Intrusion Detection,
Software

\end{nsfsummary}