 \section{Prior Results}\label{sec:PriorResults}

PI Menzies is an IEEE Fellow and has earned over \$13 million dollars in peer-reviewed competitive grants (\$6.4M from NSF, and  the rest for a variety of other government and industrial sources).
Google Scholar lists him as a top-ten researcher in many research areas including knowledge acquisition and analytics. Serving as committee chair, he has graduated 12 Ph.D. and 32 masters students (by research). He currently supervises 10 Ph.D. students at NC State. He has served as an associated editor on all the major SE journals and from 2021 will be EIC of the Automated Software Engineering journal. 


 We include below some notes  on some of his most recent NSF grants.

% PI Menzies is a co-PI with    Dr. Laurie Williams  working  on \underline{(a)}~CCF-1909516, 2019-2022, \$499,998; \underline{(b)}~``SHF: Small: Detecting the 1\%: Growing the Science of Vulnerability Detection''; \underline{(c)}~The {\bf intellectual merit} of that work was to explore characteristics of vulnerabilities with a focus on those that pose the highest security risk. The {\bf broader impact} of that work was to improve the ability of practitioners to produce secure software products so that people can rely upon computer systems to perform critical functions and to process, store, and communicate sensitive information securely. \underline{(d)}~That work has generated   journal articles (one at TSE), ICSE publications and one paper under review (at EMSE) \cite{yu2019improving,shu2019improved,elder2021structuring,shu2020omni}. \underline{(e)}~Data from that work is housed at the SEACRAFT publicly accessible repository~\cite{seacraft}. That work funded two Ph.D.s at NCSU. \underline{(f)} N/A.

PI Menzies worked on \underline{(a)}~CCF-1302216, 2013-2107, \$271,553; \underline{(b)}~``SHF: Medium: Collaborative: Transfer Learning in Software Engineering''; \underline{(c)}~The {\bf intellectual merit} of that work was to
define novel methods for sharing data, many of which were the precursor to the methods of this proposal.  That work generated the publications  \underline{(d)}~\cite{krishna2018bellwethers,peters2015lace2,he13,Me17,fu2016tuning,krishna2017learning,krishna2020whence} concerning prediction and planning methods.
The {\bf broader impact} of that work was to
enable a new kind of open science-- one where all data is routinely shared and is capable of building effective models no matter if it is obfuscated for security purposes.
The methods of this project, while targeted at software engineering, could also be applied to any other data-intensive field. \underline{(e)}~Data from that work is now housed in the two
publicly accessible repository\footnote{
{\bf github.com/rshu/Adversarial-Evasion-Defense}}. That work  funded two Ph.D.s at NCSU. \underline{(f)}
N/A. 


% One related data mining grant is \underline{(a)}~OAC-1931425, 2019-2022, \$592,129; \underline{(b)}~``Elements: Can Empirical SE be Adapted to Computational Science?''; \underline{(c)}~The {\bf intellectual merit} of that work was to create a workbench containing methods adapted from empirical software engineering, that would help bridge the skill gap via automatic agents by suggesting to developers when they should investigate or redo part of their code. The {\bf broader impact} is to reduce the associated cost (time, money, etc.) required to handle many of the large and more tedious aspects of software development. \underline{(d)}~That work generated one journal paper (at TSE), an MSR conference
% paper and two other papers under conference review~\cite{agrawal2018better,tu2021mining,tu2020changing}. \underline{(e)}~Data from that work is housed at the SEACRAFT publicly accessible repository~\cite{menzies2017seacraft}. That work funded one Ph.D. at NCSU. \underline{(f)} N/A.

Another relevant research grant is 
\underline{(a)}~OAC-1826574, 2018-2018, 
\$124,628.00;
\underline{(b)}~
``EAGER: Empirical Software Engineering for Computational Science'';
\underline{(c)}~ 
The {\bf intellectual merit} of that work was to
conduct initial explorations into novel methods for adapting SE methods to computational science.
That work lead to the curious
result that, in many ways,
the computational scientists are better
at managing their development cycle
than many SE projects~\cite{tu2020changing}. Whenever
we found good enough data to compare the results
seen in open source and computational
science projects, we often find higher productivity
values (and faster debugging) in computational science
than in software engineering. 
\underline{(d)}~That work generated 
one journal paper (at TSE'21),
one conference paper (at MSR'21) and another journal publication under review~\cite{Ling21}.
 \underline{(e)}~Data from that work is now housed at the SEACRAFT publicly accessible repository~\cite{menzies2017seacraft}. That work  funded one Ph.D. at NCSU. 
 \underline{(f)} N/A.  
 
 
 \newpage

% See also 
% \underline{(a)}~CCF-1703487; 2017-2021,  \$898,349.00; 
% \underline{(b)}~SHF: Medium: Scalable Holistic Autotuning for Software Analytics; 
% \underline{(c)} The {\bf intellectual merit}
% of that work  was that there exist previously unexplored ``short-cuts'' in the search space of control parameters of data miners. The {\bf broader impact}
% was that better learners could be created
% automatically via ``hyperparameter optimizers''
% that exploited those short-cuts. This in term meant
% that anywhere these learners were deployed, they could
% be deployed again with {\em greater} effect. 
% \underline{(d)}~This work generated five journal
% articles at TSE, EMSE, IEEE Software,  
% and one other article currently under review at EMSE~\cite{yedida2021simple,agrawal2020simpler,Yedida21,DBLP:journals/ese/YangCYYM21,menzies2021shockingly,xia2019sequential};  \underline{(e)}~Data from that work is now housed at the SEACRAFT publicly accessible repository~\cite{menzies2017seacraft}. That work  funded two Ph.D. at NCSU.  \underline{(f)}~N/A.
 

  